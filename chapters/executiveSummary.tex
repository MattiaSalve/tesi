\selectlanguage{italian}
\chapter{Executive Summary}
\fancyhead[L]{Executive Summary}

\begin{multicols}{2}
  Il seguente report tratta del progetto di consulenza svolto da me e l'ingegner Claudio Carbonaro di Acquaforte S.R.L. per Montagna S.R.L.

	L'analisi svolta presso l'azienda ha identificato due problematiche principali: l'assenza di un ruolo di gestione delle commesse e un controllo qualità inefficace. Queste criticità sono emerse chiaramente nell'organigramma aziendale, in cui il primo ruolo è completamente assente e il secondo è mal definito. Per affrontare queste problematiche, è stato creato un nuovo funzionigramma con una chiara attribuzione delle responsabilità e che sia coerente con gli obiettivi strategici dell'azienda, in particolare cercando di migliorarne l'efficienza e il coordinamento tra le varie funzioni.

	La struttura funzionale dell'azienda è stata mantenuta per la sua efficienza, con l'introduzione di due nuove unità organizzative: gestione delle commesse e gestione della qualità.
  Sono state introdotte due funzioni core: operations (divisa in logistica, produzione e manutenzione) e innovation (comprendente progettazione e area commerciale), oltre a una funzione amministrativa che include controllo e finanza, risorse umane, gestione delle strutture, acquisti e IT. Gestione qualità e gestione delle commesse, sono state separate per dare loro maggiore peso e importanza.

	Per quanto riguarda la gestione delle commesse, è stato adottato l'approccio "visible planning" sviluppato da JMAC. Questo metodo, adattato alle esigenze specifiche di Montagna, combina strumenti di project management con una forte enfasi sulla comunicazione e collaborazione tra i membri del team. È stato ritenuto che questa metodologia sia coerente con gli obiettivi e i vincoli dell'azienda, che è una piccola impresa di 35 dipendenti e ha un particolare bisogno di promuovere la collaborazione tra le aree aziendali piuttosto che un sistema rigido di project management.

	È stata poi condotta un'analisi dei consuntivi, che ha rivelato che la distribuzione delle ore tra le diverse attività è simile sia per le commesse profittevoli che per quelle non profittevoli. Tuttavia, la mancanza di standardizzazione e la variabilità delle competenze tra i lavoratori più esperti e i nuovi assunti sono state identificate come criticità che riducono l'efficienza aziendale, evidenziando la necessità di introdurre delle tecniche di standardizzazione delle competenze. A tal proposito è stato suggerito di introdurre delle JES (Job Element Sheet) per formalizzare le "best practices" e migliorare il lavoro degli operati, sviluppando le competenze dei dipendenti di conseguenza.

	Infine, è stata valutata la possibilità di adottare moduli ERP per migliorare la raccolta e la qualità dei dati sui tempi di lavorazione. Nonostante questa soluzione possa aiutare l'azienda, si è ritenuto prioritario concentrarsi sul miglioramento delle logiche gestionali e delle abitudini aziendali prima di investire in nuovi software.

	Le raccomandazioni proposte mirano a migliorare la gestione delle commesse tramite un approccio snello ispirato dalle logiche giapponesi, il controllo qualità e la standardizzazione delle competenze con l'obiettivo finale di aumentare l'efficienza e la redditività dell'azienda Montagna, aumentandone il margine e rendendola più competitiva in un mercato sempre più complesso.

\end{multicols}
