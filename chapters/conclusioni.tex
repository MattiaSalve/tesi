\selectlanguage{italian}
\chapter{Conclusioni}
\fancyhead[L]{Conclusioni}

    \section{Contributo del laureando}
\begin{multicols}{2}
Il project work è stato svolto principalmente in presenza da me, sotto la guida dell'ingegner Carbonaro. In particolare, Claudio ha svolto un giorno a settimana in azienda, mentre io mi son presentato tre volte a settimana. Due terzi del tempo, ho quindi svolto indipendentemente il lavoro concorato con Claudio durante gli incontri. 

Il coordinamento con il tutor si è svolto principalmente durante il viaggio in macchina, durante il quale scambiavamo idee e io lo aggiornavo sull'andamento del progetto e le novità dell'azienda. 

	In particolare, il coordinamento con Claudio è stato fondamentale per quanto riguarda quegli argomenti che sono stati poco affrontati durante il percorso di studi, come gli approcci agili.
    La letteratura sul lean manufacturing infatti offre interessanti spunti di riflessione su come un approccio agile alla gestione dei progetti non sia utile soltanto nell'ambito dello sviluppo software, contesto in cui è nato il manifesto agile, ma anche in ambito manifatturiero per quelle aziende che, come Montagna si ritrovano a dover gestire progetti regolarmente.
    I principi alla base delle logiche lean, come la riduzione degli sprechi e l'ottimizzazione dei processi, sono infatti la base per raggiungere un'efficienza e una flessibilità tali da permettere a un'azienda ETO di avere la flessibilità necessaria a sviluppare prodotti custom per ogni ordine pur mantenendo prezzi competitivi e qualità elevata.

	La metodologia del visible planning, ispirata ai principi del lean, rappresenta un distacco significativo nel campo della gestione dei progetti rispetto alle classiche tecniche a cascata comuni in occidente.
    L'approccio, orientato alla trasparenza, alla condivisione delle informazioni e alla risoluzione dei problemi, si inserisce perfettamente nel contesto operativo di Montagna, offrendo un quadro strutturato ed efficace per affrontare le sfide quotidiane, pur mantenendo la flessibilità tipica delle piccole e medie imprese italiane.

	Pur distaccandosi dalle metodologie affrontate durate i corsi, ho trovato estremamente interessante la tecnica proposta da Claudio, che mi ha permesso venire a conoscenza di nuovi metodi di cui non avevo mai sentito parlare. In particolare ho trovato interessanti due cose: prima di tutto queste tecniche sono, a mio parere, perfette per le piccole imprese che non hanno a disposizione il budget per software che supportino una tradizionale gestione dei progetti. La seconda cosa è che Claudio mi ha mostrato molti esempi di grandi azienda che utilizzano queste tecniche per gestire progetti molto complessi, che dimostrano la flessibilità e scalabilità di questo approccio.

    Durante i giorni in cui Claudio non era presente ho supportato il progetto comprendendo meglio il contesto aziendale, parlando con il direttore generale Domenico a lungo, introducendo le proposte di Claudio e cercando di preparare le persone per far sì che il cambio di filosofia di gestione del progetto fosse accettato da tutti, in primis dalla direzione aziendale, che dovrà supportare la transizione a queste nuove metodologie.

    Inoltre, i giorni in assenza di Claudio sono stati usati per l'analisi di dati, conducendo un'analisi più quantitativa, cercando di comprendere gli sprechi che portano a una riduzione dei margini aziendali, supportando così l'aumento dell'efficienza dovuto alla gestione della commessa con un miglioramento riguardante gli aspetti più quantitativi, come inefficienze di tempi o di materiali dovuti alle non qualità. 

	Per quanto riguarda lo sviluppo del funzionigramma, è risultata fondamentale la conoscenza di macro e microsttura appresa nel corso di "Gestione e Organizzazione Aziendale". Questo mi ha permesso di confrontarmi con Claudio e supportarlo dandogli degli spunti di riflessione, permettendoci di analizzare criticamente le proposte fatte, ottimizzando ulteriormente la nuova struttura organizzativa in funzione delle esigenze dell'azienda.

	Le conoscenze di R sviluppate nel corso di "Business Data Analytics" sono invece state di vitale importanza nella parte di analisi dei dati svolta, dandomi uno strumento molto più potente di Excel permettendomi di iterare lo stesso programma su ogni file con lo stesso formato. Questo mi ha permesso, in circa un giorno di programmazione di fare analisi su qualunque consuntivo presente, passato o futuro dell'azienda, risparmiando ore che sarebbero state passate a copiare e incollare le stesse formule su tutti i file già creati.

\section{Obiettivi conseguiti, limiti e possibili sviluppi futuri}
	Quando ho iniziato a lavorare al progetto per Montagna, questo era appena iniziato. Questo vuol dire che per fortuna sono riuscito a seguirne lo svolgimento fin dall'inizio.
    Purtroppo però, in parte per il carico di lavoro in azienda, in parte perché il progetto è lungo e deve essere implementato assicurandosi che ci sia commitment da tutti gli attori in azienda, l'implementazione di ciò che è stato trattato nel presente report è iniziata da poco.
    Questo vuol dire che purtroppo, a questo documento mancano i dati per quantificare i miglioramenti introdotti dalle azioni proposte.
    Claudio ha infatti ancora alcune giornate da passare in azienda per svolgere le prime riunioni di kickoff dei prossimi progetti e definire i prossimi interventi da compiere.
    È stata tuttavia formalizzata la proposta per la gestione della commessa, che è stata approvata dalla direzione aziendale.
    L'approccio agile è infatti stato accolto come ideale per cercare di migliorare proprio gli aspetti più critici per l'azienda.
    Ora come ora si sta aspettando l'inizio di un nuovo progetto che sia sufficientemente importante e complesso da poter essere utilizzato come progetto di kickoff per la metodologia proposta. 

	Per gli sviluppi futuri del progetto, è essenziale garantire un forte impegno da parte della dirigenza e dei dipendenti di Montagna. L'implementazione delle proposte delineate richiederà un processo graduale e iterativo, in cui l'azienda modificherà ulteriormente le metodologie proposte per allinearle al meglio con le proprie esigenze. Sono convinto che i dirigenti si impegneranno per far si che il progetto vada a buon fine, e che col necessario sforzo di tutti i dipendenti le logiche introdotte potranno portare a un miglioramento delle prestazioni aziendali, in particolare grazie alla maggiore interazione tra le aree che cambierà completamente come queste collaborano attualmente.

    Allo stesso modo, il nuovo funzionigramma è stato accettato dai vertici aziendali, che hanno compreso la coerenza con gli obiettivi strategici dell'azienda. Non è tuttavia stato possibile monitorare i miglioramenti che questo cambio ha portato in azienda in quanto il vertice aziendale non ha ancora definito le risorse che andranno a ricoprire i nuovi ruoli.
    Purtroppo, come già scritto in precedenza il periodo è particolarmente impegnativo per l'azienda e questo ha l'impatto principale proprio sul vertice, che al momento è estremamte concentrato sul condurre il core business dell'azienda, e quindi il completamento dell'organigramma è stato posto in secondo piano. 

	L'introduzione del team di controllo qualità sarà un'ulteriore area di miglioramento che è stata definita ma non completamente ultimata e che quindi attualmente non ha dati quantitativi per documentare il cambio che questa operazione ha portato in azienda. 

    Purtroppo, la criticità principale incontrata durante il progetto di consulenza è stato proprio il periodo di carico di lavoro estremo sotto cui si trova l'azienda, che ha rallentato il conseguimento degli obiettivi prefissati.

\section{Ringraziamenti}
	Ci tengo a ringraziare prima di tutto Claudio Carbonaro, con cui da subito si è creato un rapporto di fiducia che ha portato a una comunicazione onesta che ci ha permesso di collaborare per sfruttare al meglio la mia presenza in azienda. Ritengo infatti che uno dei valori aggiunti che ho portato oltre al lavoro svolto è che sono stato in azienda più spesso di Claudio, e quindi ho avuto una prospettiva più chiara delle dinamiche aziendali che ho potuto trasmettergli quando ci vedevamo. Ritengo inoltre fondamentale ringraziarlo per tutto quello che ho appreso grazie a lui: per quanto all'inizio mi sia dovuto adattare a logiche che non avevo mai incontrato durante i corsi, ritengo che proprio questo sia ciò di cui posso fare più tesoro dalla mia esperienza, che mi ha permesso di scoprire aspetti di cui non sarei mai venuto a conoscenza svolgendo un tirocinio basato su cose già studiate.

	Voglio poi estendere i miei ringraziamenti alla direzione di Montagna e a tutti i dipendenti in generale, con particolare calore verso Domenico ed Elisabetta con cui ho lavorato a stretto contatto. Fin da subito infatti mi son sentito benvenuto nell'azienda, creando un ambiente lavorativo in cui mi son davvero sentito parte di un gruppo.
\end{multicols}
