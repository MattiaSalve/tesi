\chapter{Presentazione dell'Impresa}
\fancyhead[L]{Presentazione dell'Impresa}

\selectlanguage{italian}
\section{Introduzione dell'azienda}
\begin{multicols}{2}
	Montagna S.R.L. è un’azienda locata a Rogno (BG) specializzata nella realizzazione di carpenterie per impianti di pretrattamento e verniciatura industriale, con focus principale nel settore automobilistico. Fondata come attività artigianale nel settore della lavorazione del ferro alla fine degli anni sessanta, l'azienda è cresciuta durante gli anni settanta fino a diventare una piccola industria.

	Oggi, Montagna S.R.L. conta 35 dipendenti suddivisi nelle tre aree operative: un ufficio tecnico che si occupa della progettazione delle commesse, un’area produttiva responsabile della produzione e un ufficio acquisti che si occupa dell’approvvigionamento dei materiali e altre operazioni contabili. L’azienda attualmente vanta un fatturato di circa cinque milioni di euro, che porta però a un utile alquanto basso, in media circa pari al 3\% del fatturato, mettendo l'impresa a rischio.

	Per migliorare la propria redditività, Montagna S.R.L. ha deciso di avvalersi della consulenza di Claudio Carbonaro di Acquaforte S.R.L., azienda presso la quale sto svolgendo il mio tirocinio.

	La direzione aziendale è composta da tre soci operativi e due non operativi. I tre soci operativi sono:
	\begin{itemize}
		\item Domenico, che si occupa della preventivazione, dei rapporti coi clienti e della direzione generale dell'azienda
		\item Elisabetta, responsabile acquisti e della contabilità
		\item Diego, responsabile della produzione
	\end{itemize}

	Montagna conduce il proprio business principalmente secondo una logica Engineer to Order, occupandosi di tutte le fasi del processo produttivo, dalla progettazione alla spedizione al cliente. Tuttavia, l'azienda gestisce anche alcuni ordini secondo la logica Purchase to Order (PTO), in cui il cliente si occupa della progettazione e Montagna della produzione.
\end{multicols}

\section{Problematiche principali}
\subsection{Gestione della commessa}
\begin{multicols}{2}
	Per via del proprio posizionamento sul modello di Wortmann, Montagna si trova in una situazione di elevata complessità di preventivazione e ha la necessità di gestire commesse della durata di diversi mesi, che coinvolgono tutte le funzioni aziendali. Proprio qui nascono le prime criticità individuate nell'azienda, che non è dotata di logiche di gestione della commessa, il che porta a problemi di coordinamento e a una mancanza del controllo avanzamento. Questo peggiora i margini di commessa e allunga i tempi di consegna percepiti dai clienti.

	In particolare, i problemi riguardanti la gestione della commessa nascono già dalla ricezione dell'ordine: la scarsa integrazione tra le aree aziendali infatti porta a una difficoltà nella stima dei tempi necessari alla realizzazione del prodotto, nonché dei carichi di lavoro sulle diverse funzioni. Attualmente, l'azienda non segue una logica cliente fornitore, il che significa che il reparto produzione non impone scadenze all'ufficio tecnico per la consegna dei disegni. Questa situazione è aggravata dall'assenza di pianificazione e definizione di scadenze per ogni commessa, portando anche il direttore di produzione a non avere una chiara visione del carico di lavoro del proprio reparto.

	L'assenza di un project manager è un altro problema significativo, che si traduce nella mancanza di una qualunque forma di monitoraggio dell'avanzamento della commessa. È infatti comune che l'azienda si accorga di essere in ritardo su un progetto quando è ormai troppo tardi per implementare delle soluzioni sostenibili, costringendola a ricorrere a subfornitura in quantità superiore rispetto a quella preventivata o all'utilizzo di straordinari, rischiando comunque di completare i lavori in ritardo.

	Oltre al controllo dei tempi, manca anche un efficace controllo dei costi: l'azienda infatti al momento scopre se ha fatto profitti su una commessa solo a posteriori, confrontando il consuntivo con il preventivo. Questo rende difficile comprendere le ragioni dei margini inferiori rispetto alle previsioni e impedisce di implementare azioni migliorative.
\end{multicols}

\subsection{Qualità}
\begin{multicols}{2}
	Un altro aspetto critico per quanto riguarda la soddisfazione delle esigenze del cliente e i margini di commessa è la mancanza totale di dati sulle non conformità e problemi di qualità. Secondo l'area produttiva infatti, l'azienda opera senza scarti, rilavorazioni e problemi di qualità. Sono tuttavia soventi i reclami da parte dei clienti, che portano a dover sistemare i problemi segnalati nel cantiere di assemblaggio, aumentando significativamente i costi associati alla non conformità.

	In particolare, in azienda esiste una tendenza a cercare di scaricare la colpa sugli altri, nascondendo gli errori commessi o cercando giustificazioni. Questo avviene nonostante la direzione sia interessata alla raccolta dei dati solo a fini di miglioramento del servizio offerto e non abbia mai incolpato nessuno per aver commesso errori. Questa cultura aziendale ostacola l'identificazione degli errori e la risoluzione dei problemi qualitativi incontrati.

	Montagna ha formalmente una funzione aziendale responsabile della qualità, ma i tentativi di implementare azioni di monitoraggio e miglioramento hanno incontrato resistenza da alcuni attori in azienda. Questo rende la funzione qualità solo formale e non operativa.

	La mancata raccolta di dati sulla qualità impedisce di individuare rapidamente i difetti e di analizzare le cause che portano alla produzione di pezzi non conformi. Se i difetti venissero trovati in azienda infatti sarebbe possibile capire le cause alla base della produzione da del pezzo da rilavorare o addirittura buttare. L’individuazione tardiva dei difetti rende più difficile comprendere che cosa è andato storto in fase di progettazione o produzione.
\end{multicols}

\subsection{Scarsa comunicazione interfunzionale}
\begin{multicols}{2}
	Ad oggi la comunicazione in azienda è carente: le aree funzionali comunicano infatti quasi esclusivamente tramite le interazioni tra i direttori, limitando le interazioni tra operatori ai momenti in cui emergono problemi sull’operato di un’altra area.
	Questo vuol dire che in azienda poche persone sono al corrente di quello che succede su una commessa, portando a pensare per compartimenti stagni e a una mancanza di interesse e motivazione verso gli obiettivi aziendali in generale.

	Un problema significativo è il distacco tra l’area produttiva e l’ufficio tecnico, che si manifesta a livello di percezione del lavoro svolto. L’area produttiva spesso percepisce di essere l’unica a lavorare duramente, senza comprendere le problematiche affrontate dai tecnici. Questo porta a un ulteriore distacco tra le due aree, entrambe responsabili dello sviluppo del prodotto.

	Migliorare l’interazione delle due potrebbe quindi portare non solo a una migliore integrazione e quindi maggiore efficienza aziendale, ma anche alla creazione di un ambiente lavorativo più appagante e rilassante, in cui il lavoratore si sente parte di un gruppo solido con cui può parlare apertamente dei problemi che percepisce in sul lavoro.

\end{multicols}
\newpage

\subsection{Sovraccarico di lavoro}
\begin{multicols}{2}
  
  
\end{multicols}

\section{Obiettivi del project work}
\begin{multicols}{2}
	L’obiettivo principale del progetto di consulenza è quello di riportare l’azienda in sicurezza aumentando il margine. Per fare ciò è stato ritenuto opportuno analizzare le problematiche precedentemente citate per cercare di comprendere quali sono le principali cause scatenanti di esse, cercando di proporre soluzioni implementabili in un’azienda delle dimensioni di Montagna.
\end{multicols}
